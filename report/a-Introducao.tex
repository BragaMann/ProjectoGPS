\chapter{Introdução}

\hspace{5mm} No âmbito da unidade curricular de Gestão de Processo de Software, pertencente ao perfil de Engenharia de Software do 4ºano do Mestrado Integrado em Engenharia Informática, foi proposto aos alunos que construíssem um programa com o intuito de encontrar \textit{code smell's} e que, desta forma, pusessem à prova as boas práticas das linguagens orientadas aos objetos. Um \textit{code smell} é uma característica no código fonte de um determinado programa, que poderá servir de indicador para problemas mais graves. Portanto, pode-se assim compreender o interesse de desenvolver e de enveredar por este projecto.
\par O trabalho proposto indica que este projecto seja desenvolvido em C#. Contudo, foi permitido que viesse a ser realizado na linguagem de programação Java.
\par O desenvolvimento deste projecto é compreendido por duas partes distintas. A primeira consiste num levantamento de \textit{code smells} e de bons costumes de programação na área de programação orientada aos objectos da linguagem Java. A segunda consiste na construção de um programa ("tool") que identifique/recolha esses indicadores.

