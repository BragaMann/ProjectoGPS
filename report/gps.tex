1 - long method 
2 - comentários a meio do método
3 - large class
 - normas de organização
 4 - toString(), equals(), clone()         bom hab
 5 - contrutores (vazio, parâmetros?)      bom hab
 6 - variáveis privadas.                    bom hab/innapropriate int
 7 - gets e sets -- Verificamos inexistencia.  bom hab/ inapro...
 8 - retornar e receber a interface mais genérica possível (Collection ou Map) ---- findall pela negativa.          bom hab/change preventer
 9 - verificar uso de herança - notificação.    bom hab/change preventer
10 - ciclo infinitos ex: while(true) ...        bom hab
11 - uso de variáveis primitivas.               bom hab/primitive obss
12	-identificar variáveis com 1 caracter.      bom hab
13- variaveis Final -- avisar que n se deve usar em demasia (mais de 5).  bom hab
14 - Todos os metodos têm ter exceções.       bom hab
15 - Garantir Classe tem nome maiusculo / nome classe = nome ficheiro. bom hab


Bloaters:
1     ✓ 
3     ✓ 

Dispensables:
2     ✓ 


Bons Hábitos:
4      ✓   \
              fundidos, mas depois apresentar o código separadamente
5       ✓  / 

6       ✓ 
7       ✓ 
8       ✓ 
9      



Feitos:

Filipe: 8✓ ,1✓ ,14,
Ricardo: 3✓ ,13,4✓ 
José: 11,2✓ ,5✓ 
Luis Braga: 6✓ ,12 ✓ ,9✓ 
Luis Martins: 10✓ ,7✓ ,15





































retirados:

\par Como o nome indica, quase de forma auto-explicativa, são classes que possuem demasiado acoplamento entre si. Como seria de esperar esta dependência é origem de impasses em caso de uma inevitável e obrigatória mudança.







